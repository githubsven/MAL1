\documentclass[12pt]{article}
\usepackage{hyperref}
\usepackage[affil-it]{authblk}
\usepackage{etoolbox}
\usepackage{lmodern}
\usepackage{amsmath}            % Fancy Mathematical symbols

\makeatletter
\patchcmd{\@maketitle}{\LARGE \@author}{\fontsize{12}{12}\selectfont\@author}{}{}
\makeatother
\renewcommand\Authfont{\fontsize{13}{14.4}\selectfont}
\renewcommand\Affilfont{\fontsize{9}{10.8}\itshape}

\title{Practical 1 - Reinforcement Learning \\
	\large Multi-agent learning 2017-2018}
\author{-STUDENT NAMES AND IDS HERE-}
\date{\today}
\begin{document}
\maketitle

\section*{Problem 1}\label{sec:p1}

\begin{enumerate}
	\item[a)] 						% TODO: uitwerken
	There are a few advantages for using a learning parameter $\alpha$ instead of averaging the rewards.
	The main advantage is that an agent will continue learning from any newly obtained value:
	assume an $\alpha = 0.05$ then each new value will count for $0.05 = 5\%$ towards the new expected value, regardless of how many actions were completed.
	When averaging the rewards the weights of new values will change, depending on their occurence.
	The reward of each action after five rewards will count for $1/n = 1/5 = 20\%$.
	In much later runs the weights per reward are different; old values have the same weight as newer values.
	Table \ref{table:P1-rewardWeights} shows how much each reward weighs after a number of runs.
	In a changing environment this advantage is much clearer:
	having a learning parameter $\alpha$ is much more flexible than not having one.

	We compared using a learning rate of $\alpha$ we found that the system converged much quicker than when averaging the rewards.
	The comparison was done using the optimistic initial values algorithm.
	The values used were:  nrAgents $= 10$, nrMachines $=3$ and initialValues $=5$.
	The $\alpha$ parameter varied ($\alpha = 0.05, 0.1, 0.25, 0.75$ and $1$).

	We ran the optimistic starting values algorithm five times per reward function.
	This was done for at most $1000$ ticks or until the amount of agents per slot machine did not change for 50 ticks.
	The amount of ticks before the algorithm stopped was noted and can be found in table \ref{table:P1-comparisson}.
	We found that the averaging algorithm never converged during the tests.
	When using a learning parameter it converged every time.

	Using these tests we concluded that having a learning parameter $\alpha$ helps with the rate of convergeance.
	Having a higher $\alpha$ results in faster converging strategies for the agents.


% De tests:
a = 0.05, 0.1,0.25, 0.5 ,0.75, 1.0
nrA = 10
nrM = 3
initVals = 5

	% SLides lecture 3, slide 172/174
\end{enumerate}

\section*{Problem 2}
\begin{enumerate}
	\item[a)]
	$\alpha = 0.10$
	nrAgents $= 10$, nrMachines $=3$ and initialValues $=5$
	And the initial value varied ($0$,$1$,$2$,$5$,$10$)

	The complete test data can be found in the added files.




	\item[b)]
\end{enumerate}

\section*{Problem 3}

\begin{enumerate}
	\item[a)]


	$\alpha = 0.10$
	nrAgents $= 10$, nrMachines $=3$ and initialValues $=5$
	And the $\epsilon$ varied ($0$,$1$,$2$,$5$,$10$)

	\item[b)]
\end{enumerate}

\section*{Problem 4}

\begin{enumerate}
	\item[a)]
	\item[b)]
\end{enumerate}

\section*{Problem 5}




\begin{table}[]
\centering
\begin{tabular}{|l|l|l|l|l|l|l|}
\hline
Algorithm       & Test 1                   & Test 2 & Test 3 & Test 4 & Test 5 & Average  \\ \hline
$\alpha = 0.05$ & $550$         & $365$    & $291$    & $351$    & $511$    & $413.6$ 		\\ \hline
$\alpha = 0.10$ & $430$ 				& $191$    & $372$    & $549$    & $182$    & $344.8$ 		\\ \hline
$\alpha = 0.25$ & $133$         & $62$     & $164$    & $178$    & $202$    & $147.8$ 		\\ \hline
$\alpha = 0.75$ & $61$          & $73$     & $50$     & $25$     & $128$    & $67.4$      \\ \hline
$\alpha = 1.00$ & $19$          & $21$     & $26$     & $101$    & $81$     & $49.6$      \\ \hline
Averaging       & $1000$        & $1000$   & $1000$   & $1000$   & $1000$   & $1000$      \\ \hline
\end{tabular}
\caption{Ticks required till conversion per algorithm as described in \ref{sec:p1} }
\label{table:P1-comparisson}
\end{table}


\begin{table}[]
\centering
\begin{tabular}{|l|l|l|l|l|l|l|}
\hline
Initial Value & Test 1 & Test 2 & Test 3 & Test 4 & Test 5 & Average \\ \hline
$0$       & $1$      & $1$      & $1$      & $1$      & $1$      & $1$       \\ \hline
$1$       & $351$    & $120$    & $148$    & $413$    & $187$    & $243.8$   \\ \hline
$2$       & $324$    & $316$    & $155$    & $504$    & $190$    & $297.8$   \\ \hline
$5$       & $298$    & $302$    & $237$    & $315$    & $282$    & $286.8$   \\ \hline
$10$      & $459$    & $246$    & $194$    & $359$    & $232$    & $298$     \\ \hline
\end{tabular}
\caption{The rate of convergeance with differing initial values}
\label{my-label}
\end{table}

% Extra sources:
\section{Sources}
http://users.isr.ist.utl.pt/~mtjspaan/readingGroup/ProofQlearning.pdf

2a:
\begin{table}[]
\centering
\begin{tabular}{|l|l|l|l|l|l|}
\hline
Initial Values: & $0$ & $1$   & $2$   & $5$   & $10$  \\ \hline
                & $2$ & $153$ & $199$ & $273$ & $207$ \\
                & $2$ & $255$ & $252$ & $216$ & $372$ \\
                & $2$ & $214$ & $137$ & $332$ & $291$ \\
                & $2$ & $340$ & $321$ & $481$ & $395$ \\
                & $2$ & $184$ & $274$ & $205$ & $315$ \\
                & $2$ & $221$ & $-48$ & $336$ & $299$ \\
                & $2$ & $319$ & $286$ & $255$ & $325$ \\
                & $2$ & $95$  & $501$ & $294$ & $313$ \\
                & $2$ & $-48$ & $339$ & $187$ & $446$ \\
                & $2$ & $268$ & $295$ & $173$ & $-48$ \\
                & $2$ & $240$ & $131$ & $227$ & $484$ \\
                & $2$ & $-48$ & $265$ & $448$ & $353$ \\
                & $2$ & $170$ & $245$ & $321$ & $171$ \\
                & $2$ & $244$ & $173$ & $316$ & $-48$ \\
                & $2$ & $170$ & $223$ & $316$ & $383$ \\
                & $2$ & $593$ & $167$ & $250$ & $369$ \\
                & $2$ & $243$ & $437$ & $344$ & $172$ \\
                & $2$ & $251$ & $207$ & $267$ & $371$ \\
                & $2$ & $166$ & $416$ & $300$ & $408$ \\
                & $2$ & $375$ & $296$ & $272$ & $329$ \\
                & $2$ & $314$ & $195$ & $368$ & $224$ \\
                & $2$ & $202$ & $147$ & $420$ & $299$ \\
                & $2$ & $194$ & $199$ & $243$ & $180$ \\
                & $2$ & $296$ & $244$ & $267$ & $308$ \\
                & $2$ & $219$ & $346$ & $448$ & $245$ \\
                & $2$ & $192$ & $-48$ & $223$ & $525$ \\
                & $2$ & $245$ & $243$ & $297$ & $318$ \\
                & $2$ & $103$ & $210$ & $246$ & $370$ \\
                & $2$ & $388$ & $282$ & $292$ & $316$ \\
                & $2$ & $224$ & $293$ & $421$ & $253$ \\
                & $2$ & $301$ & $548$ & $298$ & $221$ \\
                & $2$ & $344$ & $246$ & $-48$ & $320$ \\
                & $2$ & $219$ & $153$ & $-48$ & $242$ \\
                & $2$ & $343$ & $261$ & $188$ & $371$ \\
                & $4$ & $133$ & $348$ & $415$ & $251$ \\
                & $2$ & $235$ & $289$ & $131$ & $293$ \\
                & $2$ & $320$ & $276$ & $270$ & $271$ \\
                & $0$ & $180$ & $394$ & $238$ & $303$ \\
                & $2$ & $144$ & $394$ & $202$ & $337$ \\
                & $2$ & $245$ & $272$ & $249$ & $374$ \\
                & $2$ & $-48$ & $248$ & $169$ & $362$ \\
                & $3$ & $410$ & $466$ & $447$ & $224$ \\
                & $2$ & $301$ & $125$ & $-48$ & $226$ \\
                & $2$ & $174$ & $242$ & $363$ & $271$ \\
                & $2$ & $218$ & $375$ & $245$ & $370$ \\
                & $2$ & $222$ & $295$ & $225$ & $-48$ \\
                & $6$ & $146$ & $245$ & $291$ & $316$ \\
                & $2$ & $204$ & $442$ & $373$ & $395$ \\
                & $2$ & $245$ & $248$ & $227$ & $252$ \\
                & $2$ & $317$ & $231$ & $316$ & $337$ \\
                & $2$ & $392$ & $219$ & $321$ & $348$ \\
                & $2$ & $223$ & $290$ & $369$ & $295$ \\
                & $2$ & $250$ & $149$ & $225$ & $373$ \\
                & $2$ & $194$ & $370$ & $173$ & $274$ \\
                & $2$ & $349$ & $121$ & $268$ & $329$ \\
                & $2$ & $320$ & $242$ & $367$ & $382$ \\
                & $2$ & $176$ & $132$ & $245$ & $246$ \\
                & $2$ & $290$ & $281$ & $298$ & $348$ \\
                & $3$ & $128$ & $248$ & $271$ & $275$ \\
                & $2$ & $363$ & $228$ & $224$ & $300$ \\
                & $2$ & $291$ & $-48$ & $223$ & $236$ \\
                & $3$ & $252$ & $394$ & $192$ & $324$ \\
                & $2$ & $269$ & $318$ & $370$ & $245$ \\
                & $2$ & $250$ & $240$ & $270$ & $-48$ \\
                & $2$ & $269$ & $279$ & $-48$ & $218$ \\
                & $2$ & $252$ & $250$ & $-48$ & $149$ \\
                & $2$ & $-48$ & $241$ & $263$ & $319$ \\
                & $3$ & $168$ & $299$ & $420$ & $322$ \\
                & $2$ & $180$ & $245$ & $227$ & $245$ \\
                & $3$ & $178$ & $201$ & $440$ & $397$ \\
                & $2$ & $240$ & $243$ & $267$ & $293$ \\
                & $2$ & $333$ & $172$ & $351$ & $250$ \\
                & $2$ & $322$ & $368$ & $204$ & $294$ \\
                & $3$ & $395$ & $152$ & $366$ & $371$ \\
                & $2$ & $567$ & $244$ & $223$ & $246$ \\
                & $2$ & $273$ & $244$ & $-48$ & $444$ \\
                & $3$ & $271$ & $278$ & $400$ & $258$ \\
                & $2$ & $370$ & $390$ & $331$ & $406$ \\
                & $2$ & $170$ & $144$ & $228$ & $424$ \\
                & $2$ & $153$ & $500$ & $271$ & $271$ \\
                & $4$ & $290$ & $271$ & $289$ & $250$ \\
                & $2$ & $150$ & $224$ & $225$ & $238$ \\
                & $3$ & $-48$ & $195$ & $270$ & $252$ \\
                & $2$ & $265$ & $290$ & $368$ & $296$ \\
                & $2$ & $143$ & $227$ & $277$ & $397$ \\
                & $2$ & $146$ & $439$ & $315$ & $293$ \\
                & $2$ & $242$ & $226$ & $274$ & $395$ \\
                & $2$ & $183$ & $264$ & $521$ & $228$ \\
                & $2$ & $148$ & $187$ & $243$ & $168$ \\
                & $2$ & $369$ & $438$ & $227$ & $190$ \\
                & $2$ & $269$ & $322$ & $319$ & $367$ \\
                & $3$ & $248$ & $245$ & $519$ & $351$ \\
                & $2$ & $298$ & $373$ & $265$ & $267$ \\
                & $2$ & $247$ & $171$ & $252$ & $518$ \\
                & $2$ & $221$ & $392$ & $312$ & $266$ \\
                & $2$ & $149$ & $249$ & $205$ & $274$ \\
                & $2$ & $245$ & $271$ & $318$ & $308$ \\
                & $2$ & $294$ & $198$ & $372$ & $799$ \\
                & $2$ & $-48$ & $149$ & $-48$ & $413$ \\
                & $2$ & $192$ & $265$ & $249$ & $412$
\end{tabular}
\caption{My caption}
\label{Complete test results per  starting values; \alpha = 0.10, nrAgents =10, nrMachines = 3}
\end{table}

\end{document}
